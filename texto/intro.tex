%%=============================================================
%% Início da Introdução
%%=============================================================
\setcounter{page}{9}
\newgeometry{top=3cm,left=3cm,right=2cm,bottom=2cm}

\chapter*{INTRODUÇÃO}
\addcontentsline{toc}{chapter}{INTRODUÇÃO}
\label{introducao}
\thispagestyle{empty}
\setlength{\baselineskip}{1.5\baselineskip}

Há 50 anos atrás, Thomas Kuhn escrevia um livro que pretendia reposicionar a importância da história para a imagem da ciência do seu tempo. 
Esse livro, \emph{A Estrutura da Revoluções Científicas} \citeyearpar{KUHN1962}, mais tarde, foi percebido como um dos golpes 
finais em outro grupo de grande expressão na filosofia da ciência, o positivismo lógico, um golpe duro, em particular, para as concepções de filosofia da ciência carnapianas.%
    \footnote{\citet[§ 1]{FRIEDMAN1999} considera Kuhn, em conjunção com Quine, os responsáveis pelo ``desmantelamento oficial'' do positivismo lógico
     no período que fica delimitado entre a publicação de ``Dois Dogmas do Empirismo'' em \citeyearpar{QUINE1951} e
      \emph{A Estrutura das Revoluções Científicas} em \citeyearpar{KUHN1962} [doravante \emph{Estrutura}].
       \citet[p. 347]{RICHARDSON2007b} adiciona outros projetos alternativos que também comprometeram a liderança do positivismo 
       lógico na filosofia da ciência, como \citet{POPPER1959}, \citet{SELLARS1963}, \citet{POLANYI1958}, \citet{HANSON1958} e \citet{SUPPE1977}, pálidos, 
       contudo, assevera, comparados a \citet{KUHN2006}. Também, embora não de forma inteiramente explícita, o próprio \citet[p. 27]{KUHN2009} 
       sugere que está refutando as concepções dos positivistas lógicos. \citet[p. 264]{REISCH1991} cita Danto \citeyearpar{DANTO1985} para 
       reiterar essa posição de Kuhn frente ao positivismo lógico. Reisch, no entanto, contesta essa interpretação comum.}

Nos últimos tempos, no entanto, esse reconhecimento vem sendo questionado por uma série de novos trabalhos que salientam convergências entre ambas perspectivas.%
    \footnote{\citet{FRIEDMAN2002}, \citet{REISCH1991}, \citet{EARMAN1993}, \citet{IRZIK1995}.}
Kuhn várias vezes remete a uma antiga historiografia da ciência, sem contudo ser específico. Diz apenas que seu livro opõe-se a uma crença bem difundida em filosofia e história da ciência, e igualmente popular entre os cientistas,  segundo a qual o progresso da ciência seria cumulativo, e que  suas conclusões colocam em causa algumas teses epistemológicas comuns na filosofia da ciência da primeira metade do século vinte, como a distinção entre contextos de descoberta e justificação e a separação entre ciência e metafísica. Em vista disso, o livro de Kuhn foi recebido como tendo como tendo como alvo o positivismo lógico e sua concepção de ciência. 

Mais tarde, entretanto, o próprio \citet[p. 368]{KUHN2006} chegou a admitir que na época em que escreveu a \emph{Estrutura} mantinha ainda uma ``opinião cotidiana do positivismo lógico'' e que foi contra ela que reagiu. Confessou também considerar-se ``moderadamente irresponsável'' por não ter conhecido melhor seu alvo, elogiando o artigo de Irzik e Grünberg \citeyearpar{IRZIK1995}, recém publicado, que descrevia paralelos entre sua posição e a de Carnap.%
	 \footnote{Ver também \citet[p. 359]{RICHARDSON2007b}.} Esclarece Kuhn, 

\begin{quote}

        Noto que, naquilo que já foi dito, deixei de lado algo que deveria ser incluído: a questão de onde eu tirei a imagem contra a qual me rebelava na \emph{Estrutura}. Isso é, em si, uma história estranha e não inteiramente boa. Não inteiramente boa no sentido de que me dou conta, em retrospectiva, de que fui moderadamente irresponsável. Como falei, eu fiquei muito interessado, tomei um interesse real pela filosofia em meu ano de calouro e não tive, naquela ocasião, a oportunidade de praticá-la --- pelo menos não no início [\ldots] E comecei a ler o que achei que fosse filosofia da ciência --- parecia a coisa natural para se ler. E li coisas como \emph{Knowledge of the External World} [Conhecimento do mundo exterior]%
	 \footnote{\citet{RUSSELL1914}.}, de Bertrand Russell, e um bom número outras obras meio populares, meio filosóficas; li alguma coisa de Von Mises; certamente li \emph{Logic of Modern Physics} [Lógica da física moderna]%
	 \footnote{\citet{BRIDGMAN1927}.}, de Bridgman; li algo de Philipp Frank; li um pouco de Carnap, mas não o Carnap que as pessoas mais tarde apontaram como aquele que tem reais paralelos comigo. Vocês sabem, esse artigo que apareceu recentemente%
	 \footnote{\citet{IRZIK1995}.}. É um artigo muito bom. Já confessei, com grande embaraço, o fato de que eu não o conhecia [esse Carnap]. Por outro lado, também é verdade que, se eu tivesse sabido dele, se tivesse me enfronhado literatura, naquele nível, provavelmente nunca teria escrito a \emph{Estrutura}. E a visão que emerge na \emph{Estrutura} não é a mesma visão de Carnap, mas é interessante que vindo de pólos parcialmente diferentes\ldots Carnap, permanecendo dentro da tradição, tenha sido levado a isso; eu já havia me rebelado e chegado a isso vindo de outra direção e, em todo caso, permanecíamos diferentes. Mas esse era o estado de coisas na minha mente em que tive essa experiência de ter sido chamado para trabalhar no curso de Conant. E era contra esse tipo de imagem cotidiana do positivismo lógico --- eu nem mesmo pensei nisso como empirismo lógico por algum tempo ---, foi contra isso que eu reagi quando examinei meus primeiros casos em história\ldots ~\citep[pp. 367--368]{KUHN2006}%
	 \footnote{One thing I realize I left out before, that should be filled in, and that is the question as to where I got the picture that I was rebelling against in The Structure of Scientific Revolutions. And that's itself a strange and not altogether good story. Not altogether good in the sense that I realize in retrospect that I was reasonably irresponsible. I had been, as I'd said, vastly interested, caught a real interest in philosophy in my freshman year, and then had no opportunity to pursue it initially, at least. [\ldots] I started reading what I took to be philosophy of science --- it seemed the natural place to be reading. And I read things like Bertrand Russell's \emph{Knowledge of the External World}, and quite a number of others of the quasi-popular, quasi-philosophical works; I read some von Mises; I certainly read Bridgman's \emph{Logic of Modern Physics}; I read some Philipp Frank; \emph{I read a little bit of Carnap, but not the Carnap that people later point to as the stuff that has real parallels to me}. You know this article that recently appeared. It's a very good article. I have confessed to a good deal of embarrassment about the fact that I didn't know it [the Carnap]. On the other hand, \emph{it is also the case that if I'd known about it, if I'd been into that literature at that level, I probably would never have written Structure}. And the view that emerges in Structure is not the same as the Carnap view, but it's interesting that coming from what were partially different{\ldots} Carnap staying within the tradition had been driven to this --- I had rebelled already and come to it from an other direction, and in any case we were still different. But that was the state of affairs in my mind at the time that I had this experience of being asked to work in the Conant course. \emph{And it was against that sort of everyday image of logical positivism --- I didn't even think of it as logical empiricism for a while --- it was that that I was reacting to when I saw my first examples of history}\ldots~\citep[pp. 305--306]{KUHN1995}.}

\end{quote}


