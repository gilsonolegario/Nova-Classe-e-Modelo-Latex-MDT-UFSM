%%=============================================================
%% Início da Discussão
%%=============================================================
\clearpage
\chapter*{DISCUSSÃO}
\addcontentsline{toc}{chapter}{DISCUSSÃO}
\label{discussao}
\thispagestyle{empty}
\pagenumbering{arabic}


This is the Kant lipsum generator. As any dedicated reader can clearly see, the Ideal of practical reason is a representation of, as far as I know, the things in themselves; as I have shown elsewhere, the phenomena should only be used as a canon for our understanding. The paralogisms of practical reason are what first give rise to the architectonic of practical reason. As will easily be shown in the next section, reason would thereby be made to contradict, in view of these considerations, the Ideal of practical reason, yet the manifold depends on the phenomena. Necessity depends on, when thus treated as the practical employment of the never-ending regress in the series of empirical conditions, time. Human reason depends on our sense perceptions, by means of analytic unity. There can be no doubt that the objects in space and time are what first give rise to human reason.
Let us suppose that the noumena have nothing to do with necessity, since knowledge of the Categories is a posteriori. Hume tells us that the transcendental unity of apperception can not take account of the discipline of natural reason, by means of analytic unity. As is proven in the ontological manuals, it is obvious that the transcendental unity of apperception proves the validity of the Antinomies; what we have alone been able to show is that, our understanding depends on the Categories. It remains a mystery why the Ideal stands in need of reason. It must not be supposed that our faculties have lying before them, in the case of the Ideal, the Antinomies; so, the transcendental aesthetic is just as necessary as our experience. By means of the Ideal, our sense perceptions are by their very nature contradictory.
As is shown in the writings of Aristotle, the things in themselves (and it remains a mystery why this is the case) are a representation of time. Our concepts have lying before them the paralogisms of natural reason, but our a posteriori concepts have lying before them the practical employment of our experience. Because of our necessary ignorance of the conditions, the paralogisms would thereby be made to contradict, indeed, space; for these reasons, the Transcendental Deduction has lying before it our sense perceptions. (Our a posteriori knowledge can never furnish a true and demonstrated science, because, like time, it depends on analytic principles.) So, it must not be supposed that our experience depends on, so, our sense perceptions, by means of analysis. Space constitutes the whole content for our sense perceptions, and time occupies part of the sphere of the Ideal concerning the existence of the objects in space and time in general.
As we have already seen, what we have alone been able to show is that the objects in space and time would be falsified; what we have alone been able to show is that, our judgements are what first give rise to metaphysics. As I have shown elsewhere, Aristotle tells us that the objects in space and time, in the full sense of these terms, would be falsified. Let us suppose that, indeed, our problematic judgements, indeed, can be treated like our concepts. As any dedicated reader can clearly see, our knowledge can be treated like the transcendental unity of apperception, but the phenomena occupy part of the sphere of the manifold concerning the existence of natural causes in general. Whence comes the architectonic of natural reason, the solution of which involves the relation between necessity and the Categories? Natural causes (and it is not at all certain that this is the case) constitute the whole content for the paralogisms. This could not be passed over in a complete system of transcendental philosophy, but in a merely critical essay the simple mention of the fact may suffice.
Therefore, we can deduce that the objects in space and time (and I assert, however, that this is the case) have lying before them the objects in space and time. Because of our necessary ignorance of the conditions, it must not be supposed that, then, formal logic (and what we have alone been able to show is that this is true) is a representation of the never-ending regress in the series of empirical conditions, but the discipline of pure reason, in so far as this expounds the contradictory rules of metaphysics, depends on the Antinomies. By means of analytic unity, our faculties, therefore, can never, as a whole, furnish a true and demonstrated science, because, like the transcendental unity of apperception, they constitute the whole content for a priori principles; for these reasons, our experience is just as necessary as, in accordance with the principles of our a priori knowledge, philosophy. The objects in space and time abstract from all content of knowledge. Has it ever been suggested that it remains a mystery why there is no relation between the Antinomies and the phenomena? It must not be supposed that the Antinomies (and it is not at all certain that this is the case) are the clue to the discovery of philosophy, because of our necessary ignorance of the conditions. As I have shown elsewhere, to avoid all misapprehension, it is necessary to explain that our understanding (and it must not be supposed that this is true) is what first gives rise to the architectonic of pure reason, as is evident upon close examination.
The things in themselves are what first give rise to reason, as is proven in the ontological manuals. By virtue of natural reason, let us suppose that the transcendental unity of apperception abstracts from all content of knowledge; in view of these considerations, the Ideal of human reason, on the contrary, is the key to understanding pure logic. Let us suppose that, irrespective of all empirical conditions, our understanding stands in need of our disjunctive judgements. As is shown in the writings of Aristotle, pure logic, in the case of the discipline of natural reason, abstracts from all content of knowledge. Our understanding is a representation of, in accordance with the principles of the employment of the paralogisms, time. I assert, as I have shown elsewhere, that our concepts can be treated like metaphysics. By means of the Ideal, it must not be supposed that the objects in space and time are what first give rise to the employment of pure reason.
As is evident upon close examination, to avoid all misapprehension, it is necessary to explain that, on the contrary, the never-ending regress in the series of empirical conditions is a representation of our inductive judgements, yet the things in themselves prove the validity of, on the contrary, the Categories. It remains a mystery why, indeed, the never-ending regress in the series of empirical conditions exists in philosophy, but the employment of the Antinomies, in respect of the intelligible character, can never furnish a true and demonstrated science, because, like the architectonic of pure reason, it is just as necessary as problematic principles. The practical employment of the objects in space and time is by its very nature contradictory, and the thing in itself would thereby be made to contradict the Ideal of practical reason. On the other hand, natural causes can not take account of, consequently, the Antinomies, as will easily be shown in the next section. Consequently, the Ideal of practical reason (and I assert that this is true) excludes the possibility of our sense perceptions. Our experience would thereby be made to contradict, for example, our ideas, but the transcendental objects in space and time (and let us suppose that this is the case) are the clue to the discovery of necessity. But the proof of this is a task from which we can here be absolved.
Thus, the Antinomies exclude the possibility of, on the other hand, natural causes, as will easily be shown in the next section. Still, the reader should be careful to observe that the phenomena have lying before them the intelligible objects in space and time, because of the relation between the manifold and the noumena. As is evident upon close examination, Aristotle tells us that, in reference to ends, our judgements (and the reader should be careful to observe that this is the case) constitute the whole content of the empirical objects in space and time. Our experience, with the sole exception of necessity, exists in metaphysics; therefore, metaphysics exists in our experience. (It must not be supposed that the thing in itself (and I assert that this is true) may not contradict itself, but it is still possible that it may be in contradictions with the transcendental unity of apperception; certainly, our judgements exist in natural causes.) The reader should be careful to observe that, indeed, the Ideal, on the other hand, can be treated like the noumena, but natural causes would thereby be made to contradict the Antinomies. The transcendental unity of apperception constitutes the whole content for the noumena, by means of analytic unity.
In all theoretical sciences, the paralogisms of human reason would be falsified, as is proven in the ontological manuals. The architectonic of human reason is what first gives rise to the Categories. As any dedicated reader can clearly see, the paralogisms should only be used as a canon for our experience. What we have alone been able to show is that, that is to say, our sense perceptions constitute a body of demonstrated doctrine, and some of this body must be known a posteriori. Human reason occupies part of the sphere of our experience concerning the existence of the phenomena in general.