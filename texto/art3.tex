\chapter{ARTIGO 3}
\label{artigo3}
\thispagestyle{empty}
\pagenumbering{arabic}

\begin{refsection}

\noindent \textbf{Resumo}: This is the Kant lipsum generator. As any dedicated reader can clearly see, the Ideal of practical reason is a representation of, as far as I know, the things in themselves; as I have shown elsewhere, the phenomena should only be used as a canon for our understanding. The paralogisms of practical reason are what first give rise to the architectonic of practical reason. As will easily be shown in the next section, reason would thereby be made to contradict, in view of these considerations, the Ideal of practical reason, yet the manifold depends on the phenomena. Necessity depends on, when thus treated as the practical employment of the never-ending regress in the series of empirical conditions, time. Human reason depends on our sense perceptions, by means of analytic unity. There can be no doubt that the objects in space and time are what first give rise to human reason.
Let us suppose that the noumena have nothing to do with necessity, since knowledge of the Categories is a posteriori. Hume tells us that the transcendental unity of apperception can not take account of the discipline of natural reason, by means of analytic unity. As is proven in the ontological manuals, it is obvious that the transcendental unity of apperception proves the validity of the Antinomies; what we have alone been able to show is that, our understanding depends on the Categories. It remains a mystery why the Ideal stands in need of reason. It must not be supposed that our faculties have lying before them, in the case of the Ideal, the Antinomies; so, the transcendental aesthetic is just as necessary as our experience. By means of the Ideal, our sense perceptions are by their very nature contradictory.

\mbox{}

\noindent \textbf{Palavras-chave}: Xxxxx, xxxx, xxxxxxxxxxx, xxxxxxxxxxxxxx, xxxxxxxxxxxxxx

\mbox{}
\section{Introdução}



A distinção entre contextos de justificação e descoberta, disseminada por Reichenbach em \emph{Experience and Prediction} \citeyearpar{REICHENBACH1938}, foi por algum tempo considerada iluminadora e apropriada para a filosofia da ciência, mesmo por autores que normalmente faziam oposição aos empiristas lógicos, o caso, por exemplo, de Popper.%
\footnote{\citet{SIEGEL1980} e \citet{KORDIG1978} fornecem exemplos variados, tais como \citet[p. 31]{POPPER1959}, \citet[p. 472]{FEIGL1965} que a descreve como ``terminologia amplamente aceita'', \citet[p. 19]{LEWIS1946}, \citet[pp. 502--3]{CAMPBELL1977}, \citet[p. 16]{HEMPEL1966}, \citet[p. 67-73]{SCHEFFLER1967}, \citet[pp. 357--360]{ACHINSTEIN1974}, \citet[pp. 112--114]{SALMON1967}, e \citet[pp. 110--117]{KORDIG1978}.} Não, no entanto, sem receber ataques. A crescente aparição de filósofos da ciência, nas décadas de 60 e 70, que destacavam uma perspectiva historicizada --- posteriormente reconhecida como a virada historicista da filosofia da ciência --- protagonizada por Kuhn, Lakatos e Feyerabend, proporcionou um terreno fértil para pôr em causa, entre outros problemas, essa distinção. A divisão entre contextos foi primariamente apropriada e utilizada por filósofos da ciência empiristas lógicos para qualificar a filosofia da ciência como empreendimento independente de outros possíveis ângulos de análise da ciência. A tarefa da epistemologia, diferente da tarefa da sociologia, ou da psicologia da ciência, do ponto de vista empirista lógico, precisaria, sob os custos de perder-se em mal entendidos e falsas objeções, restringir-se a tratar do conhecimento de um ``modo no qual os processos de pensamento devessem ocorrer se eles estiverem em um sistema consistente [\ldots] Epistemologia então considera um substituto lógico ao invés de processos reais'' \cite[p. 4]{REICHENBACH1938}.%
    \footnote{``\ldots in a way in which they ought to occur if they are to be ranged in a consistent system [\ldots] Epistemology thus considers a logical substitute rather than real processes''. \citep[p. 4]{REICHENBACH1938}.}%

%=================== Referências ===================
\setlength{\baselineskip}{\baselineskip}

\printbibliography[heading=bibintoc]
\end{refsection}
