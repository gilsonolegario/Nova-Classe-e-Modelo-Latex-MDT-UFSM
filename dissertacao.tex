\documentclass[diss]{UFSM}
\usepackage{estilo}
% ------------------------------ Preâmbulo ------------------------------
\universidade{Universidade Federal de Santa Maria}
\universidadeeng{Federal University of Santa Maria}
\programa{Programa de Pós-graduação em Filosofia}
\programaeng{Post-Graduate Program in Philosophy}
\centro{Centro de Ciências Sociais e Humanas}
\curso{Filosofia} 
\documento{Dissertação} % ou [Tese]
\documentoeng{Dissertation}
\nivel{Mestrado} % ou [Doutorado]
\level{Master}
\grau{Mestre} % ou [Doutor]
\titulo{Título} 
\title{Título em inglês} 
\autor{Xxxxxx Xxxxxxxx xx Xxxxx} 
\citacao{XXXXXXX XX XXXXXX, Xxxxxx}
\grauorientador{Orientador: Dr.} % títulação do orientador do trabalho
\proforientador{Xxxxxx Xxxxx Xxxxxx (UFSM)} % nome do orientador do trabalho
\citacaoorientador{Xxxxx Xxxxxx, Xxxxx}
\primeiroprofbanca{Xxxxx Xxxxx\ (UFXX)}
\segundoprofbanca{Xxxxx Xxxxxx\ (UFXX)}
\diadadefesa{20 de Março}
\defenseday{March 20\textsuperscript{st}} %data em inglês, verificar a terminação
\cidade{Santa Maria} % local da defesa
\ano{2012} % ano

\keyword{Kuhn} 
\keyword{Carnap}
\keyword{holismo}
\keyword{revoluções científicas}
\keyword{incomensurabilidade}
\keyword{frameworks linguísticos}
\keyword{léxico estruturado}

\keywordeng{Kuhn} 
\keywordeng{Carnap}
\keywordeng{holism}
\keywordeng{scientific revolutions}
\keywordeng{incomensurability}
\keywordeng{linguistic frameworks}
\keywordeng{estrutural lexicons}


\comentario{\UFSMdocumento\ apresentada ao Curso de \UFSMnivel~do \UFSMprograma, Aréa de Concentração em \UFSMcurso,\\ da \UFSMuniversidade\ (UFSM, RS),\\ como requisito parcial para obtenção do grau de \\ {\bfseries\UFSMgrau\ em \UFSMcurso}.}



%=================== Início do Documento ===================
\begin{document}
\newgeometry{top=2cm,left=2cm,right=2cm,bottom=2cm}

\setlength{\baselineskip}{1.5\baselineskip}
\pagenumbering{gobble}

%=================== Capa ===================

\capa


%=================== Folha de rosto ===================

\folharosto



%=================== Ficha catalográfica ===================
\newgeometry{top=3cm,left=2cm,right=3cm,bottom=2cm}
\makeCIP{xxxxxxxx@xxxxx.com} %email do autor
\restoregeometry



%=================== Folha de aprovação ===================
\aprovacao


%=================== Dedicatória ===================

\clearpage
\begin{flushright}
\mbox{}\vfill
{\sffamily\itshape A xxxxx xxxxx e xxxx\ldots}
\end{flushright}


%=================== Agradecimentos ===================

\chapter*{Agradecimentos}



%=================== Epígrafe ===================

\clearpage
\begin{flushright}
\mbox{}\vfill
``Was Carnap entirely wrong after all?'' --- Howard Stein
\end{flushright}


%=================== Resumo e Abstract ===================


\begin{abstract}
	This is the Kant lipsum generator.As any dedicated reader can clearly see, the Ideal of practical reason is a representation of, as far as I know, the things in themselves; as I have shown elsewhere, the phenomena should only be used as a canon for our understanding. The paralogisms of practical reason are what first give rise to the architectonic of practical reason. As will easily be shown in the next section, reason would thereby be made to contradict, in view of these considerations, the Ideal of practical reason, yet the manifold depends on the phenomena. Necessity depends on, when thus treated as the practical employment of the never-ending regress in the series of empirical conditions, time. Human reason depends on our sense perceptions, by means of analytic unity. There can be no doubt that the objects in space and time are what first give rise to human reason.
	Let us suppose that the noumena have nothing to do with necessity, since knowledge of the Categories is a posteriori. Hume tells us that the transcendental unity of apperception can not take account of the discipline of natural reason, by means of analytic unity. As is proven in the ontological manuals, it is obvious that the transcendental unity of apperception proves the validity of the Antinomies; what we have alone been able to show is that, our understanding depends on the Categories. It remains a mystery why the Ideal stands in need of reason. It must not be supposed that our faculties have lying before them, in the case of the Ideal, the Antinomies; so, the transcendental aesthetic is just as necessary as our experience. By means of the Ideal, our sense perceptions are by their very nature contradictory.
	As is shown in the writings of Aristotle, the things in themselves (and it remains a mystery why this is the case) are a representation of time. Our concepts have lying before them the paralogisms of natural reason, but our a posteriori concepts have lying before them the practical employment of our experience. Because of our necessary ignorance of the conditions, the paralogisms would thereby be made to contradict, indeed, space; for these
\end{abstract}

\begin{englishabstract}
	This is the Kant lipsum generator.As any dedicated reader can clearly see, the Ideal of practical reason is a representation of, as far as I know, the things in themselves; as I have shown elsewhere, the phenomena should only be used as a canon for our understanding. The paralogisms of practical reason are what first give rise to the architectonic of practical reason. As will easily be shown in the next section, reason would thereby be made to contradict, in view of these considerations, the Ideal of practical reason, yet the manifold depends on the phenomena. Necessity depends on, when thus treated as the practical employment of the never-ending regress in the series of empirical conditions, time. Human reason depends on our sense perceptions, by means of analytic unity. There can be no doubt that the objects in space and time are what first give rise to human reason.
	Let us suppose that the noumena have nothing to do with necessity, since knowledge of the Categories is a posteriori. Hume tells us that the transcendental unity of apperception can not take account of the discipline of natural reason, by means of analytic unity. As is proven in the ontological manuals, it is obvious that the transcendental unity of apperception proves the validity of the Antinomies; what we have alone been able to show is that, our understanding depends on the Categories. It remains a mystery why the Ideal stands in need of reason. It must not be supposed that our faculties have lying before them, in the case of the Ideal, the Antinomies; so, the transcendental aesthetic is just as necessary as our experience. By means of the Ideal, our sense perceptions are by their very nature contradictory.
	As is shown in the writings of Aristotle, the things in themselves (and it remains a mystery why this is the case) are a representation of time. Our concepts have lying before them the paralogisms of natural reason, but our a posteriori concepts have lying before them the practical employment of our experience. Because of our necessary ignorance of the conditions, the paralogisms would thereby be made to contradict, indeed, space; for these
\end{englishabstract}



%=================== Sumário - ToC ===================
\setlength{\baselineskip}{\baselineskip}
\tableofcontents
\setlength{\baselineskip}{1.5\baselineskip}




%=================== Início da inclusão dos Textos ===================
\cleardoublepage
\pagenumbering{arabic}
\setcounter{page}{9}

\include{texto/intro}

\setcounter{page}{1}
\include{texto/art1}

\setcounter{page}{1}
\include{texto/art2}

\setcounter{page}{1}
\chapter*{ARTIGO 3 - CARNAP E KUHN E A DISTINÇÃO ENTRE CONTEXTOS DE DESCOBERTA E JUSTIFICAÇÃO}
\addcontentsline{toc}{chapter}{ARTIGO 3 - CARNAP E KUHN E A DISTINÇÃO ENTRE CONTEXTOS DE DESCOBERTA E JUSTIFICAÇÃO}
\label{artigo3}
\thispagestyle{empty}
\pagenumbering{arabic}

\begin{refsection}

\noindent \textbf{Resumo}: This is the Kant lipsum generator. As any dedicated reader can clearly see, the Ideal of practical reason is a representation of, as far as I know, the things in themselves; as I have shown elsewhere, the phenomena should only be used as a canon for our understanding. The paralogisms of practical reason are what first give rise to the architectonic of practical reason. As will easily be shown in the next section, reason would thereby be made to contradict, in view of these considerations, the Ideal of practical reason, yet the manifold depends on the phenomena. Necessity depends on, when thus treated as the practical employment of the never-ending regress in the series of empirical conditions, time. Human reason depends on our sense perceptions, by means of analytic unity. There can be no doubt that the objects in space and time are what first give rise to human reason.
Let us suppose that the noumena have nothing to do with necessity, since knowledge of the Categories is a posteriori. Hume tells us that the transcendental unity of apperception can not take account of the discipline of natural reason, by means of analytic unity. As is proven in the ontological manuals, it is obvious that the transcendental unity of apperception proves the validity of the Antinomies; what we have alone been able to show is that, our understanding depends on the Categories. It remains a mystery why the Ideal stands in need of reason. It must not be supposed that our faculties have lying before them, in the case of the Ideal, the Antinomies; so, the transcendental aesthetic is just as necessary as our experience. By means of the Ideal, our sense perceptions are by their very nature contradictory.

\mbox{}

\noindent \textbf{Palavras-chave}: Carnap, Kuhn, lógica da ciência, contexto de descoberta, contexto de justificação

\mbox{}
\section*{Introdução}
\addcontentsline{toc}{section}{Introdução}  



A distinção entre contextos de justificação e descoberta, disseminada por Reichenbach em \emph{Experience and Prediction} \citeyearpar{REICHENBACH1938}, foi por algum tempo considerada iluminadora e apropriada para a filosofia da ciência, mesmo por autores que normalmente faziam oposição aos empiristas lógicos, o caso, por exemplo, de Popper.%
\footnote{\citet{SIEGEL1980} e \citet{KORDIG1978} fornecem exemplos variados, tais como \citet[p. 31]{POPPER1959}, \citet[p. 472]{FEIGL1965} que a descreve como ``terminologia amplamente aceita'', \citet[p. 19]{LEWIS1946}, \citet[pp. 502--3]{CAMPBELL1977}, \citet[p. 16]{HEMPEL1966}, \citet[p. 67-73]{SCHEFFLER1967}, \citet[pp. 357--360]{ACHINSTEIN1974}, \citet[pp. 112--114]{SALMON1967}, e \citet[pp. 110--117]{KORDIG1978}.} Não, no entanto, sem receber ataques. A crescente aparição de filósofos da ciência, nas décadas de 60 e 70, que destacavam uma perspectiva historicizada --- posteriormente reconhecida como a virada historicista da filosofia da ciência --- protagonizada por Kuhn, Lakatos e Feyerabend, proporcionou um terreno fértil para pôr em causa, entre outros problemas, essa distinção. A divisão entre contextos foi primariamente apropriada e utilizada por filósofos da ciência empiristas lógicos para qualificar a filosofia da ciência como empreendimento independente de outros possíveis ângulos de análise da ciência. A tarefa da epistemologia, diferente da tarefa da sociologia, ou da psicologia da ciência, do ponto de vista empirista lógico, precisaria, sob os custos de perder-se em mal entendidos e falsas objeções, restringir-se a tratar do conhecimento de um ``modo no qual os processos de pensamento devessem ocorrer se eles estiverem em um sistema consistente [\ldots] Epistemologia então considera um substituto lógico ao invés de processos reais'' \cite[p. 4]{REICHENBACH1938}.%
    \footnote{``\ldots in a way in which they ought to occur if they are to be ranged in a consistent system [\ldots] Epistemology thus considers a logical substitute rather than real processes''. \citep[p. 4]{REICHENBACH1938}.}%

%=================== Referências ===================

\printbibliography
\addcontentsline{toc}{chapter}{Referências Bibliográficas}
\end{refsection}


\setcounter{page}{1}
\include{texto/disc}


\setcounter{page}{1}
\include{texto/concl}

\pagenumbering{gobble}

\restoregeometry
%=================== Referências ===================

\setlength{\baselineskip}{\baselineskip}

\addcontentsline{toc}{chapter}{Referências Bibliográficas}
\printbibliography

\end{document}