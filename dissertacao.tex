\documentclass[diss]{UFSM}
\usepackage{estilo}
% ------------------------------ Preâmbulo ------------------------------
\universidade{Universidade Federal de Santa Maria}
\universidadeeng{Federal University of Santa Maria}
\programa{Programa de Pós-graduação em Filosofia}
\programaeng{Post-Graduate Program in Philosophy}
\centro{Centro de Ciências Sociais e Humanas}
\curso{Filosofia} 
\documento{Dissertação} % ou [Tese]
\documentoeng{Dissertation}
\nivel{Mestrado} % ou [Doutorado]
\level{Master}
\grau{Mestre} % ou [Doutor]
\titulo{Título} 
\title{Título em inglês} 
\autor{Xxxxxx Xxxxxxxx xx Xxxxx} 
\citacao{XXXXXXX XX XXXXXX, Xxxxxx}
\grauorientador{Orientador: Dr.} % títulação do orientador do trabalho
\proforientador{Xxxxxx Xxxxx Xxxxxx (UFSM)} % nome do orientador do trabalho
\citacaoorientador{Xxxxx Xxxxxx, Xxxxx}
\primeiroprofbanca{Xxxxx Xxxxx\ (UFXX)}
\segundoprofbanca{Xxxxx Xxxxxx\ (UFXX)}
\diadadefesa{20 de Março}
\defenseday{March 20\textsuperscript{st}} %data em inglês, verificar a terminação
\cidade{Santa Maria} % local da defesa
\ano{2012} % ano

\keyword{Kuhn} 
\keyword{Carnap}
\keyword{holismo}
\keyword{revoluções científicas}
\keyword{incomensurabilidade}
\keyword{frameworks linguísticos}
\keyword{léxico estruturado}

\keywordeng{Kuhn} 
\keywordeng{Carnap}
\keywordeng{holism}
\keywordeng{scientific revolutions}
\keywordeng{incomensurability}
\keywordeng{linguistic frameworks}
\keywordeng{estrutural lexicons}


\comentario{\UFSMdocumento\ apresentada ao Curso de \UFSMnivel~do \UFSMprograma, Aréa de Concentração em \UFSMcurso,\\ da \UFSMuniversidade\ (UFSM, RS),\\ como requisito parcial para obtenção do grau de \\ {\bfseries\UFSMgrau\ em \UFSMcurso}.}



%=================== Início do Documento ===================
\begin{document}
\newgeometry{top=2cm,left=2cm,right=2cm,bottom=2cm}

\setlength{\baselineskip}{1.5\baselineskip}
\pagenumbering{gobble}

%=================== Capa ===================

\capa


%=================== Folha de rosto ===================

\folharosto



%=================== Ficha catalográfica ===================
\newgeometry{top=3cm,left=2cm,right=3cm,bottom=2cm}
\makeCIP{xxxxxxxx@xxxxx.com} %email do autor
\restoregeometry



%=================== Folha de aprovação ===================
\aprovacao


%=================== Dedicatória ===================

\clearpage
\begin{flushright}
\mbox{}\vfill
{\sffamily\itshape A xxxxx xxxxx e xxxx\ldots}
\end{flushright}


%=================== Agradecimentos ===================

\chapter*{Agradecimentos}



%=================== Epígrafe ===================

\clearpage
\begin{flushright}
\mbox{}\vfill
``Was Carnap entirely wrong after all?'' --- Howard Stein
\end{flushright}


%=================== Resumo e Abstract ===================


\begin{abstract}
	This is the Kant lipsum generator.As any dedicated reader can clearly see, the Ideal of practical reason is a representation of, as far as I know, the things in themselves; as I have shown elsewhere, the phenomena should only be used as a canon for our understanding. The paralogisms of practical reason are what first give rise to the architectonic of practical reason. As will easily be shown in the next section, reason would thereby be made to contradict, in view of these considerations, the Ideal of practical reason, yet the manifold depends on the phenomena. Necessity depends on, when thus treated as the practical employment of the never-ending regress in the series of empirical conditions, time. Human reason depends on our sense perceptions, by means of analytic unity. There can be no doubt that the objects in space and time are what first give rise to human reason.
	Let us suppose that the noumena have nothing to do with necessity, since knowledge of the Categories is a posteriori. Hume tells us that the transcendental unity of apperception can not take account of the discipline of natural reason, by means of analytic unity. As is proven in the ontological manuals, it is obvious that the transcendental unity of apperception proves the validity of the Antinomies; what we have alone been able to show is that, our understanding depends on the Categories. It remains a mystery why the Ideal stands in need of reason. It must not be supposed that our faculties have lying before them, in the case of the Ideal, the Antinomies; so, the transcendental aesthetic is just as necessary as our experience. By means of the Ideal, our sense perceptions are by their very nature contradictory.
	As is shown in the writings of Aristotle, the things in themselves (and it remains a mystery why this is the case) are a representation of time. Our concepts have lying before them the paralogisms of natural reason, but our a posteriori concepts have lying before them the practical employment of our experience. Because of our necessary ignorance of the conditions, the paralogisms would thereby be made to contradict, indeed, space; for these
\end{abstract}

\begin{englishabstract}
	This is the Kant lipsum generator.As any dedicated reader can clearly see, the Ideal of practical reason is a representation of, as far as I know, the things in themselves; as I have shown elsewhere, the phenomena should only be used as a canon for our understanding. The paralogisms of practical reason are what first give rise to the architectonic of practical reason. As will easily be shown in the next section, reason would thereby be made to contradict, in view of these considerations, the Ideal of practical reason, yet the manifold depends on the phenomena. Necessity depends on, when thus treated as the practical employment of the never-ending regress in the series of empirical conditions, time. Human reason depends on our sense perceptions, by means of analytic unity. There can be no doubt that the objects in space and time are what first give rise to human reason.
	Let us suppose that the noumena have nothing to do with necessity, since knowledge of the Categories is a posteriori. Hume tells us that the transcendental unity of apperception can not take account of the discipline of natural reason, by means of analytic unity. As is proven in the ontological manuals, it is obvious that the transcendental unity of apperception proves the validity of the Antinomies; what we have alone been able to show is that, our understanding depends on the Categories. It remains a mystery why the Ideal stands in need of reason. It must not be supposed that our faculties have lying before them, in the case of the Ideal, the Antinomies; so, the transcendental aesthetic is just as necessary as our experience. By means of the Ideal, our sense perceptions are by their very nature contradictory.
	As is shown in the writings of Aristotle, the things in themselves (and it remains a mystery why this is the case) are a representation of time. Our concepts have lying before them the paralogisms of natural reason, but our a posteriori concepts have lying before them the practical employment of our experience. Because of our necessary ignorance of the conditions, the paralogisms would thereby be made to contradict, indeed, space; for these
\end{englishabstract}



%=================== Sumário - ToC ===================
\setlength{\baselineskip}{\baselineskip}
\tableofcontents
\setlength{\baselineskip}{1.5\baselineskip}




%=================== Início da inclusão dos Textos ===================
\cleardoublepage
\pagenumbering{arabic}
\setcounter{page}{9}

%%=============================================================
%% Início da Introdução
%%=============================================================
\setcounter{page}{9}
\newgeometry{top=3cm,left=3cm,right=2cm,bottom=2cm}

\chapter*{INTRODUÇÃO}
\addcontentsline{toc}{chapter}{INTRODUÇÃO}
\label{introducao}
\thispagestyle{empty}
\setlength{\baselineskip}{1.5\baselineskip}

Há 50 anos atrás, Thomas Kuhn escrevia um livro que pretendia reposicionar a importância da história para a imagem da ciência do seu tempo. 
Esse livro, \emph{A Estrutura da Revoluções Científicas} \citeyearpar{KUHN1962}, mais tarde, foi percebido como um dos golpes 
finais em outro grupo de grande expressão na filosofia da ciência, o positivismo lógico, um golpe duro, em particular, para as concepções de filosofia da ciência carnapianas.%
    \footnote{\citet[§ 1]{FRIEDMAN1999} considera Kuhn, em conjunção com Quine, os responsáveis pelo ``desmantelamento oficial'' do positivismo lógico
     no período que fica delimitado entre a publicação de ``Dois Dogmas do Empirismo'' em \citeyearpar{QUINE1951} e
      \emph{A Estrutura das Revoluções Científicas} em \citeyearpar{KUHN1962} [doravante \emph{Estrutura}].
       \citet[p. 347]{RICHARDSON2007b} adiciona outros projetos alternativos que também comprometeram a liderança do positivismo 
       lógico na filosofia da ciência, como \citet{POPPER1959}, \citet{SELLARS1963}, \citet{POLANYI1958}, \citet{HANSON1958} e \citet{SUPPE1977}, pálidos, 
       contudo, assevera, comparados a \citet{KUHN2006}. Também, embora não de forma inteiramente explícita, o próprio \citet[p. 27]{KUHN2009} 
       sugere que está refutando as concepções dos positivistas lógicos. \citet[p. 264]{REISCH1991} cita Danto \citeyearpar{DANTO1985} para 
       reiterar essa posição de Kuhn frente ao positivismo lógico. Reisch, no entanto, contesta essa interpretação comum.}

Nos últimos tempos, no entanto, esse reconhecimento vem sendo questionado por uma série de novos trabalhos que salientam convergências entre ambas perspectivas.%
    \footnote{\citet{FRIEDMAN2002}, \citet{REISCH1991}, \citet{EARMAN1993}, \citet{IRZIK1995}.}
Kuhn várias vezes remete a uma antiga historiografia da ciência, sem contudo ser específico. Diz apenas que seu livro opõe-se a uma crença bem difundida em filosofia e história da ciência, e igualmente popular entre os cientistas,  segundo a qual o progresso da ciência seria cumulativo, e que  suas conclusões colocam em causa algumas teses epistemológicas comuns na filosofia da ciência da primeira metade do século vinte, como a distinção entre contextos de descoberta e justificação e a separação entre ciência e metafísica. Em vista disso, o livro de Kuhn foi recebido como tendo como tendo como alvo o positivismo lógico e sua concepção de ciência. 

Mais tarde, entretanto, o próprio \citet[p. 368]{KUHN2006} chegou a admitir que na época em que escreveu a \emph{Estrutura} mantinha ainda uma ``opinião cotidiana do positivismo lógico'' e que foi contra ela que reagiu. Confessou também considerar-se ``moderadamente irresponsável'' por não ter conhecido melhor seu alvo, elogiando o artigo de Irzik e Grünberg \citeyearpar{IRZIK1995}, recém publicado, que descrevia paralelos entre sua posição e a de Carnap.%
	 \footnote{Ver também \citet[p. 359]{RICHARDSON2007b}.} Esclarece Kuhn, 

\begin{quote}

        Noto que, naquilo que já foi dito, deixei de lado algo que deveria ser incluído: a questão de onde eu tirei a imagem contra a qual me rebelava na \emph{Estrutura}. Isso é, em si, uma história estranha e não inteiramente boa. Não inteiramente boa no sentido de que me dou conta, em retrospectiva, de que fui moderadamente irresponsável. Como falei, eu fiquei muito interessado, tomei um interesse real pela filosofia em meu ano de calouro e não tive, naquela ocasião, a oportunidade de praticá-la --- pelo menos não no início [\ldots] E comecei a ler o que achei que fosse filosofia da ciência --- parecia a coisa natural para se ler. E li coisas como \emph{Knowledge of the External World} [Conhecimento do mundo exterior]%
	 \footnote{\citet{RUSSELL1914}.}, de Bertrand Russell, e um bom número outras obras meio populares, meio filosóficas; li alguma coisa de Von Mises; certamente li \emph{Logic of Modern Physics} [Lógica da física moderna]%
	 \footnote{\citet{BRIDGMAN1927}.}, de Bridgman; li algo de Philipp Frank; li um pouco de Carnap, mas não o Carnap que as pessoas mais tarde apontaram como aquele que tem reais paralelos comigo. Vocês sabem, esse artigo que apareceu recentemente%
	 \footnote{\citet{IRZIK1995}.}. É um artigo muito bom. Já confessei, com grande embaraço, o fato de que eu não o conhecia [esse Carnap]. Por outro lado, também é verdade que, se eu tivesse sabido dele, se tivesse me enfronhado literatura, naquele nível, provavelmente nunca teria escrito a \emph{Estrutura}. E a visão que emerge na \emph{Estrutura} não é a mesma visão de Carnap, mas é interessante que vindo de pólos parcialmente diferentes\ldots Carnap, permanecendo dentro da tradição, tenha sido levado a isso; eu já havia me rebelado e chegado a isso vindo de outra direção e, em todo caso, permanecíamos diferentes. Mas esse era o estado de coisas na minha mente em que tive essa experiência de ter sido chamado para trabalhar no curso de Conant. E era contra esse tipo de imagem cotidiana do positivismo lógico --- eu nem mesmo pensei nisso como empirismo lógico por algum tempo ---, foi contra isso que eu reagi quando examinei meus primeiros casos em história\ldots ~\citep[pp. 367--368]{KUHN2006}%
	 \footnote{One thing I realize I left out before, that should be filled in, and that is the question as to where I got the picture that I was rebelling against in The Structure of Scientific Revolutions. And that's itself a strange and not altogether good story. Not altogether good in the sense that I realize in retrospect that I was reasonably irresponsible. I had been, as I'd said, vastly interested, caught a real interest in philosophy in my freshman year, and then had no opportunity to pursue it initially, at least. [\ldots] I started reading what I took to be philosophy of science --- it seemed the natural place to be reading. And I read things like Bertrand Russell's \emph{Knowledge of the External World}, and quite a number of others of the quasi-popular, quasi-philosophical works; I read some von Mises; I certainly read Bridgman's \emph{Logic of Modern Physics}; I read some Philipp Frank; \emph{I read a little bit of Carnap, but not the Carnap that people later point to as the stuff that has real parallels to me}. You know this article that recently appeared. It's a very good article. I have confessed to a good deal of embarrassment about the fact that I didn't know it [the Carnap]. On the other hand, \emph{it is also the case that if I'd known about it, if I'd been into that literature at that level, I probably would never have written Structure}. And the view that emerges in Structure is not the same as the Carnap view, but it's interesting that coming from what were partially different{\ldots} Carnap staying within the tradition had been driven to this --- I had rebelled already and come to it from an other direction, and in any case we were still different. But that was the state of affairs in my mind at the time that I had this experience of being asked to work in the Conant course. \emph{And it was against that sort of everyday image of logical positivism --- I didn't even think of it as logical empiricism for a while --- it was that that I was reacting to when I saw my first examples of history}\ldots~\citep[pp. 305--306]{KUHN1995}.}

\end{quote}




\setcounter{page}{1}
\chapter{ARTIGO 1}
\label{artigo1}
\thispagestyle{empty}
\pagenumbering{arabic}
\begin{refsection}



\noindent \textbf{Abstract}: This is the Kant lipsum generator.As any dedicated reader can clearly see, the Ideal of practical reason is a representation of, as far as I know, the things in themselves; as I have shown elsewhere, the phenomena should only be used as a canon for our understanding. The paralogisms of practical reason are what first give rise to the architectonic of practical reason. As will easily be shown in the next section, reason would thereby be made to contradict, in view of these considerations, the Ideal of practical reason, yet the manifold depends on the phenomena. Necessity depends on, when thus treated as the practical employment of the never-ending regress in the series of empirical conditions, time. Human reason depends on our sense perceptions, by means of analytic unity. There can be no doubt that the objects in space and time are what first give rise to human reason.
Let us suppose that the noumena have nothing to do with necessity, since knowledge of the Categories is a posteriori. Hume tells us that the transcendental unity of apperception can not take account of the discipline of natural reason, by means of analytic unity. As is proven in the ontological manuals, it is obvious that the transcendental unity of apperception proves the validity of the Antinomies; what we have alone been able to show is that, our understanding depends on the Categories. It remains a mystery why the Ideal stands in need of reason. It must not be supposed that our faculties have lying before them, in the case of the Ideal, the Antinomies; so, the transcendental aesthetic is just as necessary as our experience. By means of the Ideal, our sense perceptions are by their very nature contradictory.
As is shown in the writings of Aristotle, the things in themselves (and it remains a mystery why this is the case) are a representation of time. Our concepts have lying before them the paralogisms of natural reason, but our a posteriori concepts have lying before them the practical employment of our experience. Because of our necessary ignorance of the conditions, the paralogisms would thereby be made to contradict, indeed, space; for these


\noindent \textbf{Keywords}: holism, scientific revolutions, incommensurability, theory-ladenness 
of observations, metaphysics

\section{Introduction}



This is the Kant lipsum generator. As any dedicated reader can clearly see, the Ideal of practical reason is a representation of, as far as I know, the things in themselves; as I have shown elsewhere, the phenomena should only be used as a canon for our understanding. The paralogisms of practical reason are what first give rise to the architectonic of practical reason. As will easily be shown in the next section, reason would thereby be made to contradict, in view of these considerations, the Ideal of practical reason, yet the manifold depends on the phenomena. Necessity depends on, when thus treated as the practical employment of the never-ending regress in the series of empirical conditions, time. Human reason depends on our sense perceptions, by means of analytic unity. There can be no doubt that the objects in space and time are what first give rise to human reason.
Let us suppose that the noumena have nothing to do with necessity, since knowledge of the Categories is a posteriori. Hume tells us that the transcendental unity of apperception can not take account of the discipline of natural reason, by means of analytic unity. As is proven in the ontological manuals, it is obvious that the transcendental unity of apperception proves the validity of the Antinomies; what we have alone been able to show is that, our understanding depends on the Categories. It remains a mystery why the Ideal stands in need of reason. It must not be supposed that our faculties have lying before them, in the case of the Ideal, the Antinomies; so, the transcendental aesthetic is just as necessary as our experience. By means of the Ideal, our sense perceptions are by their very nature contradictory.
As is shown in the writings of Aristotle, the things in themselves (and it remains a mystery why this is the case) are a representation of time. Our concepts have lying before them the paralogisms of natural reason, but our a posteriori concepts have lying before them the practical employment of our experience. Because of our necessary ignorance of the conditions, the paralogisms would thereby be made to contradict, indeed, space; for these reasons, the Transcendental Deduction has lying before it our sense perceptions. (Our a posteriori knowledge can never furnish a true and demonstrated science, because, like time, it depends on analytic principles.) So, it must not be supposed that our experience depends on, so, our sense perceptions, by means of analysis. Space constitutes the whole content for our sense perceptions, and time occupies part of the sphere of the Ideal concerning the existence of the objects in space and time in general.
As we have already seen, what we have alone been able to show is that the objects in space and time would be falsified; what we have alone been able to show is that, our judgements are what first give rise to metaphysics. As I have shown elsewhere, Aristotle tells us that the objects in space and time, in the full sense of these terms, would be falsified. Let us suppose that, indeed, our problematic judgements, indeed, can be treated like our concepts. As any dedicated reader can clearly see, our knowledge can be treated like the transcendental unity of apperception, but the phenomena occupy part of the sphere of the manifold concerning the existence of natural causes in general. Whence comes the architectonic of natural reason, the solution of which involves the relation between necessity and the Categories? Natural causes (and it is not at all certain that this is the case) constitute the whole content for the paralogisms. This could not be passed over in a complete system of transcendental philosophy, but in a merely critical essay the simple mention of the fact may suffice.
Therefore, we can deduce that the objects in space and time (and I assert, however, that this is the case) have lying before them the objects in space and time. Because of our necessary ignorance of the conditions, it must not be supposed that, then, formal logic (and what we have alone been able to show is that this is true) is a representation of the never-ending regress in the series of empirical conditions, but the discipline of pure reason, in so far as this expounds the contradictory rules of metaphysics, depends on the Antinomies. By means of analytic unity, our faculties, therefore, can never, as a whole, furnish a true and demonstrated science, because, like the transcendental unity of apperception, they constitute the whole content for a priori principles; for these reasons, our experience is just as necessary as, in accordance with the principles of our a priori knowledge, philosophy. The objects in space and time abstract from all content of knowledge. Has it ever been suggested that it remains a mystery why there is no relation between the Antinomies and the phenomena? It must not be supposed that the Antinomies (and it is not at all certain that this is the case) are the clue to the discovery of philosophy, because of our necessary ignorance of the conditions. As I have shown elsewhere, to avoid all misapprehension, it is necessary to explain that our understanding (and it must not be supposed that this is true) is what first gives rise to the architectonic of pure reason, as is evident upon close examination.
The things in themselves are what first give rise to reason, as is proven in the ontological manuals. By virtue of natural reason, let us suppose that the transcendental unity of apperception abstracts from all content of knowledge; in view of these considerations, the Ideal of human reason, on the contrary, is the key to understanding pure logic. Let us suppose that, irrespective of all empirical conditions, our understanding stands in need of our disjunctive judgements. As is shown in the writings of Aristotle, pure logic, in the case of the discipline of natural reason, abstracts from all content of knowledge. Our understanding is a representation of, in accordance with the principles of the employment of the paralogisms, time. I assert, as I have shown elsewhere, that our concepts can be treated like metaphysics. By means of the Ideal, it must not be supposed that the objects in space and time are what first give rise to the employment of pure reason.
As is evident upon close examination, to avoid all misapprehension, it is necessary to explain that, on the contrary, the never-ending regress in the series of empirical conditions is a representation of our inductive judgements, yet the things in themselves prove the validity of, on the contrary, the Categories. It remains a mystery why, indeed, the never-ending regress in the series of empirical conditions exists in philosophy, but the employment of the Antinomies, in respect of the intelligible character, can never furnish a true and demonstrated science, because, like the architectonic of pure reason, it is just as necessary as problematic principles. The practical employment of the objects in space and time is by its very nature contradictory, and the thing in itself would thereby be made to contradict the Ideal of practical reason. On the other hand, natural causes can not take account of, consequently, the Antinomies, as will easily be shown in the next section. Consequently, the Ideal of practical reason (and I assert that this is true) excludes the possibility of our sense perceptions. Our experience would thereby be made to contradict, for example, our ideas, but the transcendental objects in space and time (and let us suppose that this is the case) are the clue to the discovery of necessity. But the proof of this is a task from which we can here be absolved.
Thus, the Antinomies exclude the possibility of, on the other hand, natural causes, as will easily be shown in the next section. Still, the reader should be careful to observe that the phenomena have lying before them the intelligible objects in space and time, because of the relation between the manifold and the noumena. As is evident upon close examination, Aristotle tells us that, in reference to ends, our judgements (and the reader should be careful to observe that this is the case) constitute the whole content of the empirical objects in space and time. Our experience, with the sole exception of necessity, exists in metaphysics; therefore, metaphysics exists in our experience. (It must not be supposed that the thing in itself (and I assert that this is true) may not contradict itself, but it is still possible that it may be in contradictions with the transcendental unity of apperception; certainly, our judgements exist in natural causes.) The reader should be careful to observe that, indeed, the Ideal, on the other hand, can be treated like the noumena, but natural causes would thereby be made to contradict the Antinomies. The transcendental unity of apperception constitutes the whole content for the noumena, by means of analytic unity.
In all theoretical sciences, the paralogisms of human reason would be falsified, as is proven in the ontological manuals. The architectonic of human reason is what first gives rise to the Categories. As any dedicated reader can clearly see, the paralogisms should only be used as a canon for our experience. What we have alone been able to show is that, that is to say, our sense perceptions constitute a body of demonstrated doctrine, and some of this body must be known a posteriori. Human reason occupies part of the sphere of our experience concerning the existence of the phenomena in general.
\citet[p. 19]{LEWIS1946}


%=================== Referências ===================
\setlength{\baselineskip}{\baselineskip}
\addcontentsline{toc}{section}{Referências Bibliográficas}
\printbibliography
\end{refsection}

\setcounter{page}{1}
\chapter*{ARTIGO 2}
\addcontentsline{toc}{chapter}{ARTIGO 2}
\label{artigo2}
\thispagestyle{empty}
\pagenumbering{arabic}
\begin{refsection}


\noindent \textbf{Resumo}: This is the Kant lipsum generator. As any dedicated reader can clearly see, the Ideal of practical reason is a representation of, as far as I know, the things in themselves; as I have shown elsewhere, the phenomena should only be used as a canon for our understanding. The paralogisms of practical reason are what first give rise to the architectonic of practical reason. As will easily be shown in the next section, reason would thereby be made to contradict, in view of these considerations, the Ideal of practical reason, yet the manifold depends on the phenomena. Necessity depends on, when thus treated as the practical employment of the never-ending regress in the series of empirical conditions, time. Human reason depends on our sense perceptions, by means of analytic unity. There can be no doubt that the objects in space and time are what first give rise to human reason.
Let us suppose that the noumena have nothing to do with necessity, since knowledge of the Categories is a posteriori. Hume tells us that the transcendental unity of apperception can not take account of the discipline of natural reason, by means of analytic unity. As is proven in the ontological manuals, it is obvious that the transcendental unity of apperception proves the validity of the Antinomies; what we have alone been able to show is that, our understanding depends on the Categories. It remains a mystery why the Ideal stands in need of reason. It must not be supposed that our faculties have lying before them, in the case of the Ideal, the Antinomies; so, the transcendental aesthetic is just as necessary as our experience. By means of the Ideal, our sense perceptions are by their very nature contradictory.

\mbox{}

\noindent \textbf{Palavras-chave}: Carnap, Kuhn, frameworks linguísticos, metafísica

\mbox{}
\section*{Introdução}
\addcontentsline{toc}{section}{Introdução}	

This is the Kant lipsum generator. As any dedicated reader can clearly see, the Ideal of practical reason is a representation of, as far as I know, the things in themselves; as I have shown elsewhere, the phenomena should only be used as a canon for our understanding. The paralogisms of practical reason are what first give rise to the architectonic of practical reason. As will easily be shown in the next section, reason would thereby be made to contradict, in view of these considerations, the Ideal of practical reason, yet the manifold depends on the phenomena. Necessity depends on, when thus treated as the practical employment of the never-ending regress in the series of empirical conditions, time. Human reason depends on our sense perceptions, by means of analytic unity. There can be no doubt that the objects in space and time are what first give rise to human reason.
Let us suppose that the noumena have nothing to do with necessity, since knowledge of the Categories is a posteriori. Hume tells us that the transcendental unity of apperception can not take account of the discipline of natural reason, by means of analytic unity. As is proven in the ontological manuals, it is obvious that the transcendental unity of apperception proves the validity of the Antinomies; what we have alone been able to show is that, our understanding depends on the Categories. It remains a mystery why the Ideal stands in need of reason. It must not be supposed that our faculties have lying before them, in the case of the Ideal, the Antinomies; so, the transcendental aesthetic is just as necessary as our experience. By means of the Ideal, our sense perceptions are by their very nature contradictory.
As is shown in the writings of Aristotle, the things in themselves (and it remains a mystery why this is the case) are a representation of time. Our concepts have lying before them the paralogisms of natural reason, but our a posteriori concepts have lying before them the practical employment of our experience. Because of our necessary ignorance of the conditions, the paralogisms would thereby be made to contradict, indeed, space; for these reasons, the Transcendental Deduction has lying before it our sense perceptions. (Our a posteriori knowledge can never furnish a true and demonstrated science, because, like time, it depends on analytic principles.) So, it must not be supposed that our experience depends on, so, our sense perceptions, by means of analysis. Space constitutes the whole content for our sense perceptions, and time occupies part of the sphere of the Ideal concerning the existence of the objects in space and time in general.
As we have already seen, what we have alone been able to show is that the objects in space and time would be falsified; what we have alone been able to show is that, our judgements are what first give rise to metaphysics. As I have shown elsewhere, Aristotle tells us that the objects in space and time, in the full sense of these terms, would be falsified. Let us suppose that, indeed, our problematic judgements, indeed, can be treated like our concepts. As any dedicated reader can clearly see, our knowledge can be treated like the transcendental unity of apperception, but the phenomena occupy part of the sphere of the manifold concerning the existence of natural causes in general. Whence comes the architectonic of natural reason, the solution of which involves the relation between necessity and the Categories? Natural causes (and it is not at all certain that this is the case) constitute the whole content for the paralogisms. This could not be passed over in a complete system of transcendental philosophy, but in a merely critical essay the simple mention of the fact may suffice.
Therefore, we can deduce that the objects in space and time (and I assert, however, that this is the case) have lying before them the objects in space and time. Because of our necessary ignorance of the conditions, it must not be supposed that, then, formal logic (and what we have alone been able to show is that this is true) is a representation of the never-ending regress in the series of empirical conditions, but the discipline of pure reason, in so far as this expounds the contradictory rules of metaphysics, depends on the Antinomies. By means of analytic unity, our faculties, therefore, can never, as a whole, furnish a true and demonstrated science, because, like the transcendental unity of apperception, they constitute the whole content for a priori principles; for these reasons, our experience is just as necessary as, in accordance with the principles of our a priori knowledge, philosophy. The objects in space and time abstract from all content of knowledge. Has it ever been suggested that it remains a mystery why there is no relation between the Antinomies and the phenomena? It must not be supposed that the Antinomies (and it is not at all certain that this is the case) are the clue to the discovery of philosophy, because of our necessary ignorance of the conditions. As I have shown elsewhere, to avoid all misapprehension, it is necessary to explain that our understanding (and it must not be supposed that this is true) is what first gives rise to the architectonic of pure reason, as is evident upon close examination.
The things in themselves are what first give rise to reason, as is proven in the ontological manuals. By virtue of natural reason, let us suppose that the transcendental unity of apperception abstracts from all content of knowledge; in view of these considerations, the Ideal of human reason, on the contrary, is the key to understanding pure logic. Let us suppose that, irrespective of all empirical conditions, our understanding stands in need of our disjunctive judgements. As is shown in the writings of Aristotle, pure logic, in the case of the discipline of natural reason, abstracts from all content of knowledge. Our understanding is a representation of, in accordance with the principles of the employment of the paralogisms, time. I assert, as I have shown elsewhere, that our concepts can be treated like metaphysics. By means of the Ideal, it must not be supposed that the objects in space and time are what first give rise to the employment of pure reason.
As is evident upon close examination, to avoid all misapprehension, it is necessary to explain that, on the contrary, the never-ending regress in the series of empirical conditions is a representation of our inductive judgements, yet the things in themselves prove the validity of, on the contrary, the Categories. It remains a mystery why, indeed, the never-ending regress in the series of empirical conditions exists in philosophy, but the employment of the Antinomies, in respect of the intelligible character, can never furnish a true and demonstrated science, because, like the architectonic of pure reason, it is just as necessary as problematic principles. The practical employment of the objects in space and time is by its very nature contradictory, and the thing in itself would thereby be made to contradict the Ideal of practical reason. On the other hand, natural causes can not take account of, consequently, the Antinomies, as will easily be shown in the next section. Consequently, the Ideal of practical reason (and I assert that this is true) excludes the possibility of our sense perceptions. Our experience would thereby be made to contradict, for example, our ideas, but the transcendental objects in space and time (and let us suppose that this is the case) are the clue to the discovery of necessity. But the proof of this is a task from which we can here be absolved.
Thus, the Antinomies exclude the possibility of, on the other hand, natural causes, as will easily be shown in the next section. Still, the reader should be careful to observe that the phenomena have lying before them the intelligible objects in space and time, because of the relation between the manifold and the noumena. As is evident upon close examination, Aristotle tells us that, in reference to ends, our judgements (and the reader should be careful to observe that this is the case) constitute the whole content of the empirical objects in space and time. Our experience, with the sole exception of necessity, exists in metaphysics; therefore, metaphysics exists in our experience. (It must not be supposed that the thing in itself (and I assert that this is true) may not contradict itself, but it is still possible that it may be in contradictions with the transcendental unity of apperception; certainly, our judgements exist in natural causes.) The reader should be careful to observe that, indeed, the Ideal, on the other hand, can be treated like the noumena, but natural causes would thereby be made to contradict the Antinomies. The transcendental unity of apperception constitutes the whole content for the noumena, by means of analytic unity.
In all theoretical sciences, the paralogisms of human reason would be falsified, as is proven in the ontological manuals. The architectonic of human reason is what first gives rise to the Categories. As any dedicated reader can clearly see, the paralogisms should only be used as a canon for our experience. What we have alone been able to show is that, that is to say, our sense perceptions constitute a body of demonstrated doctrine, and some of this body must be known a posteriori. Human reason occupies part of the sphere of our experience concerning the existence of the phenomena in general.
\citet[p. 19]{LEWIS1946}

%=================== Referências ===================
\setlength{\baselineskip}{\baselineskip}
\addcontentsline{toc}{section}{Referências Bibliográficas}
\printbibliography
\end{refsection}


\setcounter{page}{1}
\chapter{ARTIGO 3}
\label{artigo3}
\thispagestyle{empty}
\pagenumbering{arabic}

\begin{refsection}

\noindent \textbf{Resumo}: This is the Kant lipsum generator. As any dedicated reader can clearly see, the Ideal of practical reason is a representation of, as far as I know, the things in themselves; as I have shown elsewhere, the phenomena should only be used as a canon for our understanding. The paralogisms of practical reason are what first give rise to the architectonic of practical reason. As will easily be shown in the next section, reason would thereby be made to contradict, in view of these considerations, the Ideal of practical reason, yet the manifold depends on the phenomena. Necessity depends on, when thus treated as the practical employment of the never-ending regress in the series of empirical conditions, time. Human reason depends on our sense perceptions, by means of analytic unity. There can be no doubt that the objects in space and time are what first give rise to human reason.
Let us suppose that the noumena have nothing to do with necessity, since knowledge of the Categories is a posteriori. Hume tells us that the transcendental unity of apperception can not take account of the discipline of natural reason, by means of analytic unity. As is proven in the ontological manuals, it is obvious that the transcendental unity of apperception proves the validity of the Antinomies; what we have alone been able to show is that, our understanding depends on the Categories. It remains a mystery why the Ideal stands in need of reason. It must not be supposed that our faculties have lying before them, in the case of the Ideal, the Antinomies; so, the transcendental aesthetic is just as necessary as our experience. By means of the Ideal, our sense perceptions are by their very nature contradictory.

\mbox{}

\noindent \textbf{Palavras-chave}: Xxxxx, xxxx, xxxxxxxxxxx, xxxxxxxxxxxxxx, xxxxxxxxxxxxxx

\mbox{}
\section{Introdução}



A distinção entre contextos de justificação e descoberta, disseminada por Reichenbach em \emph{Experience and Prediction} \citeyearpar{REICHENBACH1938}, foi por algum tempo considerada iluminadora e apropriada para a filosofia da ciência, mesmo por autores que normalmente faziam oposição aos empiristas lógicos, o caso, por exemplo, de Popper.%
\footnote{\citet{SIEGEL1980} e \citet{KORDIG1978} fornecem exemplos variados, tais como \citet[p. 31]{POPPER1959}, \citet[p. 472]{FEIGL1965} que a descreve como ``terminologia amplamente aceita'', \citet[p. 19]{LEWIS1946}, \citet[pp. 502--3]{CAMPBELL1977}, \citet[p. 16]{HEMPEL1966}, \citet[p. 67-73]{SCHEFFLER1967}, \citet[pp. 357--360]{ACHINSTEIN1974}, \citet[pp. 112--114]{SALMON1967}, e \citet[pp. 110--117]{KORDIG1978}.} Não, no entanto, sem receber ataques. A crescente aparição de filósofos da ciência, nas décadas de 60 e 70, que destacavam uma perspectiva historicizada --- posteriormente reconhecida como a virada historicista da filosofia da ciência --- protagonizada por Kuhn, Lakatos e Feyerabend, proporcionou um terreno fértil para pôr em causa, entre outros problemas, essa distinção. A divisão entre contextos foi primariamente apropriada e utilizada por filósofos da ciência empiristas lógicos para qualificar a filosofia da ciência como empreendimento independente de outros possíveis ângulos de análise da ciência. A tarefa da epistemologia, diferente da tarefa da sociologia, ou da psicologia da ciência, do ponto de vista empirista lógico, precisaria, sob os custos de perder-se em mal entendidos e falsas objeções, restringir-se a tratar do conhecimento de um ``modo no qual os processos de pensamento devessem ocorrer se eles estiverem em um sistema consistente [\ldots] Epistemologia então considera um substituto lógico ao invés de processos reais'' \cite[p. 4]{REICHENBACH1938}.%
    \footnote{``\ldots in a way in which they ought to occur if they are to be ranged in a consistent system [\ldots] Epistemology thus considers a logical substitute rather than real processes''. \citep[p. 4]{REICHENBACH1938}.}%

%=================== Referências ===================
\setlength{\baselineskip}{\baselineskip}

\printbibliography[heading=bibintoc]
\end{refsection}


\setcounter{page}{1}
\chapter{DISCUSSÃO}
\label{discussao}
\thispagestyle{empty}
\pagenumbering{arabic}


This is the Kant lipsum generator. As any dedicated reader can clearly see, the Ideal of practical reason is a representation of, as far as I know, the things in themselves; as I have shown elsewhere, the phenomena should only be used as a canon for our understanding. The paralogisms of practical reason are what first give rise to the architectonic of practical reason. As will easily be shown in the next section, reason would thereby be made to contradict, in view of these considerations, the Ideal of practical reason, yet the manifold depends on the phenomena. Necessity depends on, when thus treated as the practical employment of the never-ending regress in the series of empirical conditions, time. Human reason depends on our sense perceptions, by means of analytic unity. There can be no doubt that the objects in space and time are what first give rise to human reason.
Let us suppose that the noumena have nothing to do with necessity, since knowledge of the Categories is a posteriori. Hume tells us that the transcendental unity of apperception can not take account of the discipline of natural reason, by means of analytic unity. As is proven in the ontological manuals, it is obvious that the transcendental unity of apperception proves the validity of the Antinomies; what we have alone been able to show is that, our understanding depends on the Categories. It remains a mystery why the Ideal stands in need of reason. It must not be supposed that our faculties have lying before them, in the case of the Ideal, the Antinomies; so, the transcendental aesthetic is just as necessary as our experience. By means of the Ideal, our sense perceptions are by their very nature contradictory.
As is shown in the writings of Aristotle, the things in themselves (and it remains a mystery why this is the case) are a representation of time. Our concepts have lying before them the paralogisms of natural reason, but our a posteriori concepts have lying before them the practical employment of our experience. Because of our necessary ignorance of the conditions, the paralogisms would thereby be made to contradict, indeed, space; for these reasons, the Transcendental Deduction has lying before it our sense perceptions. (Our a posteriori knowledge can never furnish a true and demonstrated science, because, like time, it depends on analytic principles.) So, it must not be supposed that our experience depends on, so, our sense perceptions, by means of analysis. Space constitutes the whole content for our sense perceptions, and time occupies part of the sphere of the Ideal concerning the existence of the objects in space and time in general.
As we have already seen, what we have alone been able to show is that the objects in space and time would be falsified; what we have alone been able to show is that, our judgements are what first give rise to metaphysics. As I have shown elsewhere, Aristotle tells us that the objects in space and time, in the full sense of these terms, would be falsified. Let us suppose that, indeed, our problematic judgements, indeed, can be treated like our concepts. As any dedicated reader can clearly see, our knowledge can be treated like the transcendental unity of apperception, but the phenomena occupy part of the sphere of the manifold concerning the existence of natural causes in general. Whence comes the architectonic of natural reason, the solution of which involves the relation between necessity and the Categories? Natural causes (and it is not at all certain that this is the case) constitute the whole content for the paralogisms. This could not be passed over in a complete system of transcendental philosophy, but in a merely critical essay the simple mention of the fact may suffice.
Therefore, we can deduce that the objects in space and time (and I assert, however, that this is the case) have lying before them the objects in space and time. Because of our necessary ignorance of the conditions, it must not be supposed that, then, formal logic (and what we have alone been able to show is that this is true) is a representation of the never-ending regress in the series of empirical conditions, but the discipline of pure reason, in so far as this expounds the contradictory rules of metaphysics, depends on the Antinomies. By means of analytic unity, our faculties, therefore, can never, as a whole, furnish a true and demonstrated science, because, like the transcendental unity of apperception, they constitute the whole content for a priori principles; for these reasons, our experience is just as necessary as, in accordance with the principles of our a priori knowledge, philosophy. The objects in space and time abstract from all content of knowledge. Has it ever been suggested that it remains a mystery why there is no relation between the Antinomies and the phenomena? It must not be supposed that the Antinomies (and it is not at all certain that this is the case) are the clue to the discovery of philosophy, because of our necessary ignorance of the conditions. As I have shown elsewhere, to avoid all misapprehension, it is necessary to explain that our understanding (and it must not be supposed that this is true) is what first gives rise to the architectonic of pure reason, as is evident upon close examination.
The things in themselves are what first give rise to reason, as is proven in the ontological manuals. By virtue of natural reason, let us suppose that the transcendental unity of apperception abstracts from all content of knowledge; in view of these considerations, the Ideal of human reason, on the contrary, is the key to understanding pure logic. Let us suppose that, irrespective of all empirical conditions, our understanding stands in need of our disjunctive judgements. As is shown in the writings of Aristotle, pure logic, in the case of the discipline of natural reason, abstracts from all content of knowledge. Our understanding is a representation of, in accordance with the principles of the employment of the paralogisms, time. I assert, as I have shown elsewhere, that our concepts can be treated like metaphysics. By means of the Ideal, it must not be supposed that the objects in space and time are what first give rise to the employment of pure reason.
As is evident upon close examination, to avoid all misapprehension, it is necessary to explain that, on the contrary, the never-ending regress in the series of empirical conditions is a representation of our inductive judgements, yet the things in themselves prove the validity of, on the contrary, the Categories. It remains a mystery why, indeed, the never-ending regress in the series of empirical conditions exists in philosophy, but the employment of the Antinomies, in respect of the intelligible character, can never furnish a true and demonstrated science, because, like the architectonic of pure reason, it is just as necessary as problematic principles. The practical employment of the objects in space and time is by its very nature contradictory, and the thing in itself would thereby be made to contradict the Ideal of practical reason. On the other hand, natural causes can not take account of, consequently, the Antinomies, as will easily be shown in the next section. Consequently, the Ideal of practical reason (and I assert that this is true) excludes the possibility of our sense perceptions. Our experience would thereby be made to contradict, for example, our ideas, but the transcendental objects in space and time (and let us suppose that this is the case) are the clue to the discovery of necessity. But the proof of this is a task from which we can here be absolved.
Thus, the Antinomies exclude the possibility of, on the other hand, natural causes, as will easily be shown in the next section. Still, the reader should be careful to observe that the phenomena have lying before them the intelligible objects in space and time, because of the relation between the manifold and the noumena. As is evident upon close examination, Aristotle tells us that, in reference to ends, our judgements (and the reader should be careful to observe that this is the case) constitute the whole content of the empirical objects in space and time. Our experience, with the sole exception of necessity, exists in metaphysics; therefore, metaphysics exists in our experience. (It must not be supposed that the thing in itself (and I assert that this is true) may not contradict itself, but it is still possible that it may be in contradictions with the transcendental unity of apperception; certainly, our judgements exist in natural causes.) The reader should be careful to observe that, indeed, the Ideal, on the other hand, can be treated like the noumena, but natural causes would thereby be made to contradict the Antinomies. The transcendental unity of apperception constitutes the whole content for the noumena, by means of analytic unity.
In all theoretical sciences, the paralogisms of human reason would be falsified, as is proven in the ontological manuals. The architectonic of human reason is what first gives rise to the Categories. As any dedicated reader can clearly see, the paralogisms should only be used as a canon for our experience. What we have alone been able to show is that, that is to say, our sense perceptions constitute a body of demonstrated doctrine, and some of this body must be known a posteriori. Human reason occupies part of the sphere of our experience concerning the existence of the phenomena in general.


\setcounter{page}{1}
\chapter{CONCLUSÃO}
\label{conclusao}
\thispagestyle{empty}
\pagenumbering{arabic}


This is the Kant lipsum generator. As any dedicated reader can clearly see, the Ideal of practical reason is a representation of, as far as I know, the things in themselves; as I have shown elsewhere, the phenomena should only be used as a canon for our understanding. The paralogisms of practical reason are what first give rise to the architectonic of practical reason. As will easily be shown in the next section, reason would thereby be made to contradict, in view of these considerations, the Ideal of practical reason, yet the manifold depends on the phenomena. Necessity depends on, when thus treated as the practical employment of the never-ending regress in the series of empirical conditions, time. Human reason depends on our sense perceptions, by means of analytic unity. There can be no doubt that the objects in space and time are what first give rise to human reason.
Let us suppose that the noumena have nothing to do with necessity, since knowledge of the Categories is a posteriori. Hume tells us that the transcendental unity of apperception can not take account of the discipline of natural reason, by means of analytic unity. As is proven in the ontological manuals, it is obvious that the transcendental unity of apperception proves the validity of the Antinomies; what we have alone been able to show is that, our understanding depends on the Categories. It remains a mystery why the Ideal stands in need of reason. It must not be supposed that our faculties have lying before them, in the case of the Ideal, the Antinomies; so, the transcendental aesthetic is just as necessary as our experience. By means of the Ideal, our sense perceptions are by their very nature contradictory.
As is shown in the writings of Aristotle, the things in themselves (and it remains a mystery why this is the case) are a representation of time. Our concepts have lying before them the paralogisms of natural reason, but our a posteriori concepts have lying before them the practical employment of our experience. Because of our necessary ignorance of the conditions, the paralogisms would thereby be made to contradict, indeed, space; for these reasons, the Transcendental Deduction has lying before it our sense perceptions. (Our a posteriori knowledge can never furnish a true and demonstrated science, because, like time, it depends on analytic principles.) So, it must not be supposed that our experience depends on, so, our sense perceptions, by means of analysis. Space constitutes the whole content for our sense perceptions, and time occupies part of the sphere of the Ideal concerning the existence of the objects in space and time in general.
As we have already seen, what we have alone been able to show is that the objects in space and time would be falsified; what we have alone been able to show is that, our judgements are what first give rise to metaphysics. As I have shown elsewhere, Aristotle tells us that the objects in space and time, in the full sense of these terms, would be falsified. Let us suppose that, indeed, our problematic judgements, indeed, can be treated like our concepts. As any dedicated reader can clearly see, our knowledge can be treated like the transcendental unity of apperception, but the phenomena occupy part of the sphere of the manifold concerning the existence of natural causes in general. Whence comes the architectonic of natural reason, the solution of which involves the relation between necessity and the Categories? Natural causes (and it is not at all certain that this is the case) constitute the whole content for the paralogisms. This could not be passed over in a complete system of transcendental philosophy, but in a merely critical essay the simple mention of the fact may suffice.
Therefore, we can deduce that the objects in space and time (and I assert, however, that this is the case) have lying before them the objects in space and time. Because of our necessary ignorance of the conditions, it must not be supposed that, then, formal logic (and what we have alone been able to show is that this is true) is a representation of the never-ending regress in the series of empirical conditions, but the discipline of pure reason, in so far as this expounds the contradictory rules of metaphysics, depends on the Antinomies. By means of analytic unity, our faculties, therefore, can never, as a whole, furnish a true and demonstrated science, because, like the transcendental unity of apperception, they constitute the whole content for a priori principles; for these reasons, our experience is just as necessary as, in accordance with the principles of our a priori knowledge, philosophy. The objects in space and time abstract from all content of knowledge. Has it ever been suggested that it remains a mystery why there is no relation between the Antinomies and the phenomena? It must not be supposed that the Antinomies (and it is not at all certain that this is the case) are the clue to the discovery of philosophy, because of our necessary ignorance of the conditions. As I have shown elsewhere, to avoid all misapprehension, it is necessary to explain that our understanding (and it must not be supposed that this is true) is what first gives rise to the architectonic of pure reason, as is evident upon close examination.
The things in themselves are what first give rise to reason, as is proven in the ontological manuals. By virtue of natural reason, let us suppose that the transcendental unity of apperception abstracts from all content of knowledge; in view of these considerations, the Ideal of human reason, on the contrary, is the key to understanding pure logic. Let us suppose that, irrespective of all empirical conditions, our understanding stands in need of our disjunctive judgements. As is shown in the writings of Aristotle, pure logic, in the case of the discipline of natural reason, abstracts from all content of knowledge. Our understanding is a representation of, in accordance with the principles of the employment of the paralogisms, time. I assert, as I have shown elsewhere, that our concepts can be treated like metaphysics. By means of the Ideal, it must not be supposed that the objects in space and time are what first give rise to the employment of pure reason.
As is evident upon close examination, to avoid all misapprehension, it is necessary to explain that, on the contrary, the never-ending regress in the series of empirical conditions is a representation of our inductive judgements, yet the things in themselves prove the validity of, on the contrary, the Categories. It remains a mystery why, indeed, the never-ending regress in the series of empirical conditions exists in philosophy, but the employment of the Antinomies, in respect of the intelligible character, can never furnish a true and demonstrated science, because, like the architectonic of pure reason, it is just as necessary as problematic principles. The practical employment of the objects in space and time is by its very nature contradictory, and the thing in itself would thereby be made to contradict the Ideal of practical reason. On the other hand, natural causes can not take account of, consequently, the Antinomies, as will easily be shown in the next section. Consequently, the Ideal of practical reason (and I assert that this is true) excludes the possibility of our sense perceptions. Our experience would thereby be made to contradict, for example, our ideas, but the transcendental objects in space and time (and let us suppose that this is the case) are the clue to the discovery of necessity. But the proof of this is a task from which we can here be absolved.
Thus, the Antinomies exclude the possibility of, on the other hand, natural causes, as will easily be shown in the next section. Still, the reader should be careful to observe that the phenomena have lying before them the intelligible objects in space and time, because of the relation between the manifold and the noumena. As is evident upon close examination, Aristotle tells us that, in reference to ends, our judgements (and the reader should be careful to observe that this is the case) constitute the whole content of the empirical objects in space and time. Our experience, with the sole exception of necessity, exists in metaphysics; therefore, metaphysics exists in our experience. (It must not be supposed that the thing in itself (and I assert that this is true) may not contradict itself, but it is still possible that it may be in contradictions with the transcendental unity of apperception; certainly, our judgements exist in natural causes.) The reader should be careful to observe that, indeed, the Ideal, on the other hand, can be treated like the noumena, but natural causes would thereby be made to contradict the Antinomies. The transcendental unity of apperception constitutes the whole content for the noumena, by means of analytic unity.
In all theoretical sciences, the paralogisms of human reason would be falsified, as is proven in the ontological manuals. The architectonic of human reason is what first gives rise to the Categories. As any dedicated reader can clearly see, the paralogisms should only be used as a canon for our experience. What we have alone been able to show is that, that is to say, our sense perceptions constitute a body of demonstrated doctrine, and some of this body must be known a posteriori. Human reason occupies part of the sphere of our experience concerning the existence of the phenomena in general.

\pagenumbering{gobble}

\restoregeometry
%=================== Referências ===================

\setlength{\baselineskip}{\baselineskip}

\addcontentsline{toc}{chapter}{Referências Bibliográficas}
\printbibliography

\end{document}